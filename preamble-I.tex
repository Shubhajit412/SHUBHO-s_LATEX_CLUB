%%%%%%%%%%%%%%%%%%%%%%%%%%%%%%%%%%%%%
% Preamble-I
%%%%%%%%%%%%%%%%%%%%%%%%%%%%%%%%%%%%%

%%%%%%%%%%%%%%%%
% Disclaimer.
%%%%%%%%%%%%%%%%
% This is an academic document template created without any additional dependencies
% or technical requirements. Please note that this is version = v1.0 officially
% released on 28 Mar 2025, written by Shubhajit Dey (shubhajitdey412@icloud.com).

% It may be distributed and/or modified under the conditions of the LaTeX Project
% Public License, either version 1.3 of this license or (at your option) any later
% version. The latest version of this license is in http://www.latex-project.org/lppl.txt
% and version 1.3 or later is part of all distributions of LaTeX version 2003/12/01
% or later.

%------------------------------------
% Commands to use this preamble
%------------------------------------

% Keep this file in the same working directory in which your project is saved and
% insert the following command in the first line of your project. 
% %%%%%%%%%%%%%%%%%%%%%%%%%%%%%%%%%%%%%
% Preamble-I
%%%%%%%%%%%%%%%%%%%%%%%%%%%%%%%%%%%%%

%%%%%%%%%%%%%%%%
% Disclaimer.
%%%%%%%%%%%%%%%%
% This is an academic document template created without any additional dependencies
% or technical requirements. Please note that this is version = v1.0 officially
% released on 28 Mar 2025, written by Shubhajit Dey (shubhajitdey412@icloud.com).

% It may be distributed and/or modified under the conditions of the LaTeX Project
% Public License, either version 1.3 of this license or (at your option) any later
% version. The latest version of this license is in http://www.latex-project.org/lppl.txt
% and version 1.3 or later is part of all distributions of LaTeX version 2003/12/01
% or later.

%------------------------------------
% Commands to use this preamble
%------------------------------------

% Keep this file in the same working directory in which your project is saved and.
% insert the following command in the first line of your project. 
% %%%%%%%%%%%%%%%%%%%%%%%%%%%%%%%%%%%%%
% Preamble-I
%%%%%%%%%%%%%%%%%%%%%%%%%%%%%%%%%%%%%

%%%%%%%%%%%%%%%%
% Disclaimer.
%%%%%%%%%%%%%%%%
% This is an academic document template created without any additional dependencies
% or technical requirements. Please note that this is version = v1.0 officially
% released on 28 Mar 2025, written by Shubhajit Dey (shubhajitdey412@icloud.com).

% It may be distributed and/or modified under the conditions of the LaTeX Project
% Public License, either version 1.3 of this license or (at your option) any later
% version. The latest version of this license is in http://www.latex-project.org/lppl.txt
% and version 1.3 or later is part of all distributions of LaTeX version 2003/12/01
% or later.

%------------------------------------
% Commands to use this preamble
%------------------------------------

% Keep this file in the same working directory in which your project is saved and.
% insert the following command in the first line of your project. 
% %%%%%%%%%%%%%%%%%%%%%%%%%%%%%%%%%%%%%
% Preamble-I
%%%%%%%%%%%%%%%%%%%%%%%%%%%%%%%%%%%%%

%%%%%%%%%%%%%%%%
% Disclaimer.
%%%%%%%%%%%%%%%%
% This is an academic document template created without any additional dependencies
% or technical requirements. Please note that this is version = v1.0 officially
% released on 28 Mar 2025, written by Shubhajit Dey (shubhajitdey412@icloud.com).

% It may be distributed and/or modified under the conditions of the LaTeX Project
% Public License, either version 1.3 of this license or (at your option) any later
% version. The latest version of this license is in http://www.latex-project.org/lppl.txt
% and version 1.3 or later is part of all distributions of LaTeX version 2003/12/01
% or later.

%------------------------------------
% Commands to use this preamble
%------------------------------------

% Keep this file in the same working directory in which your project is saved and.
% insert the following command in the first line of your project. 
% \input{preamble-I.tex}
%%%%%%%%%%%%%%%%

\documentclass[12pt]{article}
\usepackage[margin=0.9in,a4paper]{geometry}
\setlength{\parindent}{0pt} %...........indent length for new paragraphs.
\setlength{\parskip}{5pt} %.............distance between paragraphs.
\pagenumbering{arabic} %................sets the page numbering style.

%------------------------------------
% All Packages
%------------------------------------

% math packages...
\usepackage{amsmath,amsthm,amssymb} %........all a mathematician needs.
\usepackage{mathtools} %.....................extends amsmath with extra math features.
\usepackage{yhmath} %........................adds extra math symbols (wide accents, large operators).
\usepackage{mathpartir} %....................for natural deduction-style proofs and inference rules.

% color and framing...
\usepackage[dvipsnames,table]{xcolor} %......adds color capabilities (text, tables, backgrounds).
\usepackage{tcolorbox} %.....................creates colored and customizable text boxes.
\usepackage{framed} %........................creates framed content blocks.

% lists, layouts and tables...
\usepackage{enumitem} %......................for enumeration enviro.
\usepackage{makecell} %......................allows multi-line text and formatting inside table cells.
\usepackage{multirow} %......................merges multiple rows into a single table cell.
\usepackage{multicol} %......................creates multi-column content in the document.
\usepackage{paracol} %.......................creates parallel columns (useful for multilingual texts).
\usepackage{titlesec} %......................customizes section and subsection formatting.
\usepackage{tocloft} %.......................customizes the Table of Contents (ToC) layout and formatting.
\usepackage{hyperref} %......................adds hyperlinks inside the generated pdf.

% figures and captions...
\usepackage{subfigure} %.....................adds multiple sub-figures inside a single figure environment.
\usepackage{float} %.........................controls the placement of floating environments (figures/tables).
\usepackage{tikz} %..........................creates complex graphics and diagrams directly in LaTeX.
\usepackage{subcaption} %....................similar to subfigure but with more flexible captioning.

% algorithms and pseudocode...
\usepackage{algorithmicx} %..................core package for algorithm environments.
\usepackage[ruled]{algorithm} %..............adds floating environments for algorithms (with captioning).
\usepackage{algpseudocode} %.................adds pseudocode syntax to algorithm environment.
\usepackage{algpascal} %.....................provides Pascal-style pseudocode.
\usepackage{algc} %..........................provides C-style pseudocode.

% text and symbols...
\usepackage{textcomp} %......................adds extra symbols (e.g., \textmu for µ).

% appendices and collecting...
\usepackage{appendix} %......................simplifies the creation of appendices.
\usepackage{collectbox} %....................collects and processes content inside a box.

%------------------------------------
% Theorem Environments
%------------------------------------

% under usage...
\newtheorem{euclidtheorem}{Proposition}[section]
\newtheorem{theorem}{Theorem}[section]
\newtheorem{classtheorem}{Definition}[section]
\newtheorem{gausstheorem}{Corollary}[section]
\newtheorem{eulertheorem}{Lemma}[section]

\newtheorem{challenge}[theorem]{Challenge}
\newtheorem{question}[theorem]{Question}
\newtheorem{problem}[theorem]{Problem}

\newtheorem{theorempiece}{Theorem}[theorem]
\newtheorem{classtheorempiece}{Theorem}[classtheorem]
\newtheorem{challengepiece}{Challenge}[theorem]
\newtheorem{questionpiece}{Question}[theorem]
\newtheorem{problempiece}{Problem}[theorem]


% under usage...
\renewcommand*{\theeuclidtheorem}{\arabic{section}.\arabic{subsection}.\arabic{theorem}}
\renewcommand*{\thetheorem}{\arabic{section}.\arabic{subsection}.\arabic{theorem}}
\renewcommand*{\theclasstheorem}{\arabic{section}.\arabic{subsection}.\arabic{theorem}}
\renewcommand*{\thegausstheorem}{\arabic{section}.\arabic{subsection}.\arabic{theorem}}
\renewcommand*{\theeulertheorem}{\arabic{section}.\arabic{subsection}.\arabic{theorem}}

\renewcommand*{\thechallenge}{\arabic{section}.\arabic{theorem}}
\renewcommand*{\thequestion}{\arabic{section}.\arabic{theorem}}
\renewcommand*{\theproblem}{\arabic{section}.\arabic{theorem}}
\renewcommand*{\thetheorempiece}{\arabic{section}.\arabic{theorem}.\alph{theorempiece}}
\renewcommand*{\thechallengepiece}{\arabic{section}.\arabic{theorem}.\alph{challengepiece}}
\renewcommand*{\thequestionpiece}{\arabic{section}.\arabic{theorem}.\alph{questionpiece}}
\renewcommand*{\theproblempiece}{\arabic{section}.\arabic{theorem}.\alph{problempiece}}

%------------------------------------
% Custom Symbols
%------------------------------------
\newcommand{\btr}{\blacktriangleright}
\newcommand{\notimplies}{\;\not\!\!\!\implies}
\newcommand{\rat}{\rightarrowtail}

%------------------------------------
% Colors
%------------------------------------
\definecolor{paynes.grey}{HTML}{536878}
\definecolor{wine}{HTML}{722F37}
\definecolor{dutch.white}{HTML}{efdfbb}

\colorlet{paynes.grey}{paynes.grey}
\colorlet{wine}{wine}
\colorlet{dutch.white}{dutch.white}

%------------------------------------
% Hyperlink setup
%------------------------------------
\hypersetup{
	colorlinks=true,       % false: boxed links; true: colored links
	linkcolor=wine,        % color of internal links
	citecolor=blue,        % color of links to bibliography
	filecolor=magenta,     % color of file links
	urlcolor=blue         
}

%------------------------------------
% Customize the Table of Contents 
%------------------------------------
\renewcommand{\cftsecleader}{\cftdotfill{\cftdotsep}} %.............dotted lines for sections.
\renewcommand{\cftsubsecleader}{\cftdotfill{\cftdotsep}} %..........dotted lines for subsections.
\cftsetindents{section}{0.25em}{1.8em} %............................adjust the space between section number & title.
\cftsetindents{subsection}{1.5em}{2.3em} %.........[exp]............adjust the space between subsection number & title.

%------------------------------------
% Additional Commands
%------------------------------------ 
\titleformat{\section}{\Large\bfseries\centering}{}{0em}{\underline} %......customize section header formatting to hide section numbers & other font edits

%%%%%%%%%%%%%%%%%%%%%%%%%%%%%%%%%%%%%
% Document
%%%%%%%%%%%%%%%%%%%%%%%%%%%%%%%%%%%%%
%%%%%%%%%%%%%%%%

\documentclass[12pt]{article}
\usepackage[margin=0.9in,a4paper]{geometry}
\setlength{\parindent}{0pt} %...........indent length for new paragraphs.
\setlength{\parskip}{5pt} %.............distance between paragraphs.
\pagenumbering{arabic} %................sets the page numbering style.

%------------------------------------
% All Packages
%------------------------------------

% math packages...
\usepackage{amsmath,amsthm,amssymb} %........all a mathematician needs.
\usepackage{mathtools} %.....................extends amsmath with extra math features.
\usepackage{yhmath} %........................adds extra math symbols (wide accents, large operators).
\usepackage{mathpartir} %....................for natural deduction-style proofs and inference rules.

% color and framing...
\usepackage[dvipsnames,table]{xcolor} %......adds color capabilities (text, tables, backgrounds).
\usepackage{tcolorbox} %.....................creates colored and customizable text boxes.
\usepackage{framed} %........................creates framed content blocks.

% lists, layouts and tables...
\usepackage{enumitem} %......................for enumeration enviro.
\usepackage{makecell} %......................allows multi-line text and formatting inside table cells.
\usepackage{multirow} %......................merges multiple rows into a single table cell.
\usepackage{multicol} %......................creates multi-column content in the document.
\usepackage{paracol} %.......................creates parallel columns (useful for multilingual texts).
\usepackage{titlesec} %......................customizes section and subsection formatting.
\usepackage{tocloft} %.......................customizes the Table of Contents (ToC) layout and formatting.
\usepackage{hyperref} %......................adds hyperlinks inside the generated pdf.

% figures and captions...
\usepackage{subfigure} %.....................adds multiple sub-figures inside a single figure environment.
\usepackage{float} %.........................controls the placement of floating environments (figures/tables).
\usepackage{tikz} %..........................creates complex graphics and diagrams directly in LaTeX.
\usepackage{subcaption} %....................similar to subfigure but with more flexible captioning.

% algorithms and pseudocode...
\usepackage{algorithmicx} %..................core package for algorithm environments.
\usepackage[ruled]{algorithm} %..............adds floating environments for algorithms (with captioning).
\usepackage{algpseudocode} %.................adds pseudocode syntax to algorithm environment.
\usepackage{algpascal} %.....................provides Pascal-style pseudocode.
\usepackage{algc} %..........................provides C-style pseudocode.

% text and symbols...
\usepackage{textcomp} %......................adds extra symbols (e.g., \textmu for µ).

% appendices and collecting...
\usepackage{appendix} %......................simplifies the creation of appendices.
\usepackage{collectbox} %....................collects and processes content inside a box.

%------------------------------------
% Theorem Environments
%------------------------------------

% under usage...
\newtheorem{euclidtheorem}{Proposition}[section]
\newtheorem{theorem}{Theorem}[section]
\newtheorem{classtheorem}{Definition}[section]
\newtheorem{gausstheorem}{Corollary}[section]
\newtheorem{eulertheorem}{Lemma}[section]

\newtheorem{challenge}[theorem]{Challenge}
\newtheorem{question}[theorem]{Question}
\newtheorem{problem}[theorem]{Problem}

\newtheorem{theorempiece}{Theorem}[theorem]
\newtheorem{classtheorempiece}{Theorem}[classtheorem]
\newtheorem{challengepiece}{Challenge}[theorem]
\newtheorem{questionpiece}{Question}[theorem]
\newtheorem{problempiece}{Problem}[theorem]


% under usage...
\renewcommand*{\theeuclidtheorem}{\arabic{section}.\arabic{subsection}.\arabic{theorem}}
\renewcommand*{\thetheorem}{\arabic{section}.\arabic{subsection}.\arabic{theorem}}
\renewcommand*{\theclasstheorem}{\arabic{section}.\arabic{subsection}.\arabic{theorem}}
\renewcommand*{\thegausstheorem}{\arabic{section}.\arabic{subsection}.\arabic{theorem}}
\renewcommand*{\theeulertheorem}{\arabic{section}.\arabic{subsection}.\arabic{theorem}}

\renewcommand*{\thechallenge}{\arabic{section}.\arabic{theorem}}
\renewcommand*{\thequestion}{\arabic{section}.\arabic{theorem}}
\renewcommand*{\theproblem}{\arabic{section}.\arabic{theorem}}
\renewcommand*{\thetheorempiece}{\arabic{section}.\arabic{theorem}.\alph{theorempiece}}
\renewcommand*{\thechallengepiece}{\arabic{section}.\arabic{theorem}.\alph{challengepiece}}
\renewcommand*{\thequestionpiece}{\arabic{section}.\arabic{theorem}.\alph{questionpiece}}
\renewcommand*{\theproblempiece}{\arabic{section}.\arabic{theorem}.\alph{problempiece}}

%------------------------------------
% Custom Symbols
%------------------------------------
\newcommand{\btr}{\blacktriangleright}
\newcommand{\notimplies}{\;\not\!\!\!\implies}
\newcommand{\rat}{\rightarrowtail}

%------------------------------------
% Colors
%------------------------------------
\definecolor{paynes.grey}{HTML}{536878}
\definecolor{wine}{HTML}{722F37}
\definecolor{dutch.white}{HTML}{efdfbb}

\colorlet{paynes.grey}{paynes.grey}
\colorlet{wine}{wine}
\colorlet{dutch.white}{dutch.white}

%------------------------------------
% Hyperlink setup
%------------------------------------
\hypersetup{
	colorlinks=true,       % false: boxed links; true: colored links
	linkcolor=wine,        % color of internal links
	citecolor=blue,        % color of links to bibliography
	filecolor=magenta,     % color of file links
	urlcolor=blue         
}

%------------------------------------
% Customize the Table of Contents 
%------------------------------------
\renewcommand{\cftsecleader}{\cftdotfill{\cftdotsep}} %.............dotted lines for sections.
\renewcommand{\cftsubsecleader}{\cftdotfill{\cftdotsep}} %..........dotted lines for subsections.
\cftsetindents{section}{0.25em}{1.8em} %............................adjust the space between section number & title.
\cftsetindents{subsection}{1.5em}{2.3em} %.........[exp]............adjust the space between subsection number & title.

%------------------------------------
% Additional Commands
%------------------------------------ 
\titleformat{\section}{\Large\bfseries\centering}{}{0em}{\underline} %......customize section header formatting to hide section numbers & other font edits

%%%%%%%%%%%%%%%%%%%%%%%%%%%%%%%%%%%%%
% Document
%%%%%%%%%%%%%%%%%%%%%%%%%%%%%%%%%%%%%
%%%%%%%%%%%%%%%%

\documentclass[12pt]{article}
\usepackage[margin=0.9in,a4paper]{geometry}
\setlength{\parindent}{0pt} %...........indent length for new paragraphs.
\setlength{\parskip}{5pt} %.............distance between paragraphs.
\pagenumbering{arabic} %................sets the page numbering style.

%------------------------------------
% All Packages
%------------------------------------

% math packages...
\usepackage{amsmath,amsthm,amssymb} %........all a mathematician needs.
\usepackage{mathtools} %.....................extends amsmath with extra math features.
\usepackage{yhmath} %........................adds extra math symbols (wide accents, large operators).
\usepackage{mathpartir} %....................for natural deduction-style proofs and inference rules.

% color and framing...
\usepackage[dvipsnames,table]{xcolor} %......adds color capabilities (text, tables, backgrounds).
\usepackage{tcolorbox} %.....................creates colored and customizable text boxes.
\usepackage{framed} %........................creates framed content blocks.

% lists, layouts and tables...
\usepackage{enumitem} %......................for enumeration enviro.
\usepackage{makecell} %......................allows multi-line text and formatting inside table cells.
\usepackage{multirow} %......................merges multiple rows into a single table cell.
\usepackage{multicol} %......................creates multi-column content in the document.
\usepackage{paracol} %.......................creates parallel columns (useful for multilingual texts).
\usepackage{titlesec} %......................customizes section and subsection formatting.
\usepackage{tocloft} %.......................customizes the Table of Contents (ToC) layout and formatting.
\usepackage{hyperref} %......................adds hyperlinks inside the generated pdf.

% figures and captions...
\usepackage{subfigure} %.....................adds multiple sub-figures inside a single figure environment.
\usepackage{float} %.........................controls the placement of floating environments (figures/tables).
\usepackage{tikz} %..........................creates complex graphics and diagrams directly in LaTeX.
\usepackage{subcaption} %....................similar to subfigure but with more flexible captioning.

% algorithms and pseudocode...
\usepackage{algorithmicx} %..................core package for algorithm environments.
\usepackage[ruled]{algorithm} %..............adds floating environments for algorithms (with captioning).
\usepackage{algpseudocode} %.................adds pseudocode syntax to algorithm environment.
\usepackage{algpascal} %.....................provides Pascal-style pseudocode.
\usepackage{algc} %..........................provides C-style pseudocode.

% text and symbols...
\usepackage{textcomp} %......................adds extra symbols (e.g., \textmu for µ).

% appendices and collecting...
\usepackage{appendix} %......................simplifies the creation of appendices.
\usepackage{collectbox} %....................collects and processes content inside a box.

%------------------------------------
% Theorem Environments
%------------------------------------

% under usage...
\newtheorem{euclidtheorem}{Proposition}[section]
\newtheorem{theorem}{Theorem}[section]
\newtheorem{classtheorem}{Definition}[section]
\newtheorem{gausstheorem}{Corollary}[section]
\newtheorem{eulertheorem}{Lemma}[section]

\newtheorem{challenge}[theorem]{Challenge}
\newtheorem{question}[theorem]{Question}
\newtheorem{problem}[theorem]{Problem}

\newtheorem{theorempiece}{Theorem}[theorem]
\newtheorem{classtheorempiece}{Theorem}[classtheorem]
\newtheorem{challengepiece}{Challenge}[theorem]
\newtheorem{questionpiece}{Question}[theorem]
\newtheorem{problempiece}{Problem}[theorem]


% under usage...
\renewcommand*{\theeuclidtheorem}{\arabic{section}.\arabic{subsection}.\arabic{theorem}}
\renewcommand*{\thetheorem}{\arabic{section}.\arabic{subsection}.\arabic{theorem}}
\renewcommand*{\theclasstheorem}{\arabic{section}.\arabic{subsection}.\arabic{theorem}}
\renewcommand*{\thegausstheorem}{\arabic{section}.\arabic{subsection}.\arabic{theorem}}
\renewcommand*{\theeulertheorem}{\arabic{section}.\arabic{subsection}.\arabic{theorem}}

\renewcommand*{\thechallenge}{\arabic{section}.\arabic{theorem}}
\renewcommand*{\thequestion}{\arabic{section}.\arabic{theorem}}
\renewcommand*{\theproblem}{\arabic{section}.\arabic{theorem}}
\renewcommand*{\thetheorempiece}{\arabic{section}.\arabic{theorem}.\alph{theorempiece}}
\renewcommand*{\thechallengepiece}{\arabic{section}.\arabic{theorem}.\alph{challengepiece}}
\renewcommand*{\thequestionpiece}{\arabic{section}.\arabic{theorem}.\alph{questionpiece}}
\renewcommand*{\theproblempiece}{\arabic{section}.\arabic{theorem}.\alph{problempiece}}

%------------------------------------
% Custom Symbols
%------------------------------------
\newcommand{\btr}{\blacktriangleright}
\newcommand{\notimplies}{\;\not\!\!\!\implies}
\newcommand{\rat}{\rightarrowtail}

%------------------------------------
% Colors
%------------------------------------
\definecolor{paynes.grey}{HTML}{536878}
\definecolor{wine}{HTML}{722F37}
\definecolor{dutch.white}{HTML}{efdfbb}

\colorlet{paynes.grey}{paynes.grey}
\colorlet{wine}{wine}
\colorlet{dutch.white}{dutch.white}

%------------------------------------
% Hyperlink setup
%------------------------------------
\hypersetup{
	colorlinks=true,       % false: boxed links; true: colored links
	linkcolor=wine,        % color of internal links
	citecolor=blue,        % color of links to bibliography
	filecolor=magenta,     % color of file links
	urlcolor=blue         
}

%------------------------------------
% Customize the Table of Contents 
%------------------------------------
\renewcommand{\cftsecleader}{\cftdotfill{\cftdotsep}} %.............dotted lines for sections.
\renewcommand{\cftsubsecleader}{\cftdotfill{\cftdotsep}} %..........dotted lines for subsections.
\cftsetindents{section}{0.25em}{1.8em} %............................adjust the space between section number & title.
\cftsetindents{subsection}{1.5em}{2.3em} %.........[exp]............adjust the space between subsection number & title.

%------------------------------------
% Additional Commands
%------------------------------------ 
\titleformat{\section}{\Large\bfseries\centering}{}{0em}{\underline} %......customize section header formatting to hide section numbers & other font edits

%%%%%%%%%%%%%%%%%%%%%%%%%%%%%%%%%%%%%
% Document
%%%%%%%%%%%%%%%%%%%%%%%%%%%%%%%%%%%%%
%%%%%%%%%%%%%%%%

\documentclass[12pt]{article}
\usepackage[margin=0.9in,a4paper]{geometry}
\setlength{\parindent}{0pt} %...........indent length for new paragraphs.
\setlength{\parskip}{5pt} %.............distance between paragraphs.
\pagenumbering{arabic} %................sets the page numbering style.

%------------------------------------
% All Packages
%------------------------------------

% math packages...
\usepackage{amsmath,amsthm,amssymb} %........all a mathematician needs.
\usepackage{mathtools} %.....................extends amsmath with extra math features.
\usepackage{yhmath} %........................adds extra math symbols (wide accents, large operators).
\usepackage{mathpartir} %....................for natural deduction-style proofs and inference rules.

% color and framing...
\usepackage[dvipsnames,table]{xcolor} %......adds color capabilities (text, tables, backgrounds).
\usepackage{tcolorbox} %.....................creates colored and customizable text boxes.
\usepackage{framed} %........................creates framed content blocks.

% lists, layouts and tables...
\usepackage{enumitem} %......................for enumeration enviro.
\usepackage{makecell} %......................allows multi-line text and formatting inside table cells.
\usepackage{multirow} %......................merges multiple rows into a single table cell.
\usepackage{multicol} %......................creates multi-column content in the document.
\usepackage{paracol} %.......................creates parallel columns (useful for multilingual texts).
\usepackage{titlesec} %......................customizes section and subsection formatting.
\usepackage{tocloft} %.......................customizes the Table of Contents (ToC) layout and formatting.
\usepackage{hyperref} %......................adds hyperlinks inside the generated pdf.

% figures and captions...
\usepackage{subfigure} %.....................adds multiple sub-figures inside a single figure environment.
\usepackage{float} %.........................controls the placement of floating environments (figures/tables).
\usepackage{tikz} %..........................creates complex graphics and diagrams directly in LaTeX.
\usepackage{subcaption} %....................similar to subfigure but with more flexible captioning.
\usepackage{graphicx} %......................imports and manipulates graphics and images.

% algorithms and pseudocode...
\usepackage{algorithmicx} %..................core package for algorithm environments.
\usepackage[ruled]{algorithm} %..............adds floating environments for algorithms (with captioning).
\usepackage{algpseudocode} %.................adds pseudocode syntax to algorithm environment.
\usepackage{algpascal} %.....................provides Pascal-style pseudocode.
\usepackage{algc} %..........................provides C-style pseudocode.

% text and symbols...
\usepackage{textcomp} %......................adds extra symbols (e.g., \textmu for µ).

% appendices and collecting...
\usepackage{appendix} %......................simplifies the creation of appendices.
\usepackage{collectbox} %....................collects and processes content inside a box.

%------------------------------------
% Theorem Environments
%------------------------------------

% under usage...
\newtheorem{euclidtheorem}{Proposition}[section]
\newtheorem{theorem}{Theorem}[section]
\newtheorem{classtheorem}{Definition}[section]
\newtheorem{gausstheorem}{Corollary}[section]
\newtheorem{eulertheorem}{Lemma}[section]

\newtheorem{challenge}[theorem]{Challenge}
\newtheorem{question}[theorem]{Question}
\newtheorem{problem}[theorem]{Problem}

\newtheorem{theorempiece}{Theorem}[theorem]
\newtheorem{classtheorempiece}{Theorem}[classtheorem]
\newtheorem{challengepiece}{Challenge}[theorem]
\newtheorem{questionpiece}{Question}[theorem]
\newtheorem{problempiece}{Problem}[theorem]


% under usage...
\renewcommand*{\theeuclidtheorem}{\arabic{section}.\arabic{subsection}.\arabic{theorem}}
\renewcommand*{\thetheorem}{\arabic{section}.\arabic{subsection}.\arabic{theorem}}
\renewcommand*{\theclasstheorem}{\arabic{section}.\arabic{subsection}.\arabic{theorem}}
\renewcommand*{\thegausstheorem}{\arabic{section}.\arabic{subsection}.\arabic{theorem}}
\renewcommand*{\theeulertheorem}{\arabic{section}.\arabic{subsection}.\arabic{theorem}}

\renewcommand*{\thechallenge}{\arabic{section}.\arabic{theorem}}
\renewcommand*{\thequestion}{\arabic{section}.\arabic{theorem}}
\renewcommand*{\theproblem}{\arabic{section}.\arabic{theorem}}
\renewcommand*{\thetheorempiece}{\arabic{section}.\arabic{theorem}.\alph{theorempiece}}
\renewcommand*{\thechallengepiece}{\arabic{section}.\arabic{theorem}.\alph{challengepiece}}
\renewcommand*{\thequestionpiece}{\arabic{section}.\arabic{theorem}.\alph{questionpiece}}
\renewcommand*{\theproblempiece}{\arabic{section}.\arabic{theorem}.\alph{problempiece}}

%------------------------------------
% Custom Symbols
%------------------------------------
\newcommand{\btr}{\blacktriangleright}
\newcommand{\notimplies}{\;\not\!\!\!\implies}
\newcommand{\rat}{\rightarrowtail}

%------------------------------------
% Colors
%------------------------------------
\definecolor{paynes.grey}{HTML}{536878}
\definecolor{wine}{HTML}{722F37}
\definecolor{dutch.white}{HTML}{efdfbb}

\colorlet{paynes.grey}{paynes.grey}
\colorlet{wine}{wine}
\colorlet{dutch.white}{dutch.white}

%------------------------------------
% Hyperlink setup
%------------------------------------
\hypersetup{
	colorlinks=true,       % false: boxed links; true: colored links
	linkcolor=wine,        % color of internal links
	citecolor=blue,        % color of links to bibliography
	filecolor=magenta,     % color of file links
	urlcolor=blue         
}

%------------------------------------
% Customize the Table of Contents 
%------------------------------------
\renewcommand{\cftsecleader}{\cftdotfill{\cftdotsep}} %.............dotted lines for sections.
\renewcommand{\cftsubsecleader}{\cftdotfill{\cftdotsep}} %..........dotted lines for subsections.
\cftsetindents{section}{0.25em}{1.8em} %............................adjust the space between section number & title.
\cftsetindents{subsection}{1.5em}{2.3em} %.........[exp]............adjust the space between subsection number & title.

%------------------------------------
% Additional Commands
%------------------------------------ 
\titleformat{\section}{\Large\bfseries\centering}{}{0em}{\underline} %......customize section header formatting to hide section numbers & other font edits
\graphicspath{{Images/}} %..................................................directory to keep all the images used in the project

%%%%%%%%%%%%%%%%%%%%%%%%%%%%%%%%%%%%%
% Document
%%%%%%%%%%%%%%%%%%%%%%%%%%%%%%%%%%%%%
